\documentclass[11pt]{article}

\usepackage[utf8]{inputenc}
\usepackage[nottoc,numbib]{tocbibind}
\usepackage[a4paper, margin=2cm]{geometry}
\usepackage{amsmath}
\usepackage{amssymb}
\usepackage{graphicx}

% From https://tex.stackexchange.com/questions/116534/lstlisting-line-wrapping
\usepackage{listings}
\usepackage{amsmath}
\usepackage{xcolor}
\lstset{
  columns=fullflexible,
  frame=single,
  numbers=left,
  breaklines=true,
  postbreak=\mbox{\textcolor{red}{$\hookrightarrow$}\space},
}

\graphicspath{{./images/}}

\title{EMBS Assessment 2}
\author{\input{./exam_number.txt}}
\date{June 2023}

\begin{document}

\begin{titlepage}
\maketitle
Word Count: \input{./word_count.txt}
\tableofcontents
\end{titlepage}

\section{System Design}

% design process:
% - write proof-of-concept code in (ideally) platform-agnostic C on personal computer
% - write test suite to confirm correctness on laptop where it's much easier to run and capture results
% - create project for hardware, hook in ethernet and HDMI minimum viable products
% - copy in proof-of-concept code and perform minimal required process to hook up to hardware IO
% - continued relative separation of IO interfacing code and solving algorithm allows quick iteration of algo code on laptop where it's easier to do

% - recursive solving method
% - expand from bottom left corner, shift entire puzzle up or right when can't place any further tiles
% - terminate after a certain number of solutions have been found, or iteration is complete

% primary data structures:
% - tile, in final form just a 4-byte array
% - field, consists primarily of an array of tiles and a mapping of "board" locations to indexes into tile array
%   - index logic implemented is stride-flattened

% no hardware parallelism

\newpage
\section{Parallelism in the System}

% no hardware parallelism, but in theory:
% - IO operations and high level decision making run on ARM
% - control flow also on arm, ie. recursive calls
% - repetitive data-crunching operations (eg. rotating tiles) performed by hardware
% - could perhaps develop custom IP cores for control flow

\newpage
\section{Evaluation and Testing}

% methods:
% - manual running on target hardware and visually confirming functionality
%   - debug over serial with custom printing if necessary
%     - show screenshots of serial/photos of hdmi
% - test suite on laptop using open source unit testing framework unity
% - benchmarking tools on laptop:
%   - hyperfine for statistically robust performance numbers
%   - flamegraph allows insight into what functions are taking the most time

% show graphs from data
% gather data on laptop performance and show
% extrapolate parallel time reduction with parallelism on laptop?

\end{document}
